\documentclass{classrep}
\usepackage[utf8]{inputenc}

\author{
  \studentinfo{Agata Jasionowska}{229726}
}

\title{Laboratorium \ppauza Lista 1}
\begin{document}

\maketitle
% to jest komentarz
\% <- wydrukowano procent!!!!!

\section{Zadanie 1}
	\subsection{Opis problemu}
		Napisanie w języku \texttt{Julia} implementacji czterech algorytmów obliczających iloczyn skalarny dwóch 
		zadanych wektorów.
	\subsection{Wnioski}
		Treść zadania podaje prawidłowy wynik równy $-1.00657107000000e{-11}$.
		Wynik iloczynu jest zbliżony do $0.0$, co oznacza, że wektory prostopadłe (ortogonalne !!!), które mają tendencję do generowania dużych błędów
		względnych.

\end{document}
